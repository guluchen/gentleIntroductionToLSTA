\section{Probabilistic vs Quantum Systems}

\begin{wrapfigure}[15]{R}{0.5\textwidth}
\includegraphics[width=0.48\textwidth]{Figures/MChains/Mchain} 
\caption{(a) A Markov chain. (b) Its adjacency matrix.}
\label{MChain:fig}
\end{wrapfigure}

In probabilistic systems, there exists an inherent uncertainty in our knowledge of the physical state.  
%
Moreover, state transitions occur according to probabilistic laws.  
%
This implies that the evolution of such systems is governed by rules specifying the likelihood of transitions between states.  
%
A classical model capturing this behaviour is the \emph{Markov chain}—an automaton whose transitions are labelled with \emph{weights} \cite{DBLP:books/daglib/0020348,DBLP:conf/fossacs/PiribauerB19}.
%
A weight is a real number between $0$ and $1$, indicating the probability of moving from the source state to the target state.  
%
The probabilities of all transitions emanating from a given state sum to one.  

We can represent a Markov chain using its adjacency matrix $M$, which is a square matrix with real-valued entries.  
%
Each row and column of $M$ corresponds to a state, and the entry $M[q', q]$ denotes the weight of the transition from state $q$ to state $q'$.  
%
By construction, the sum of the entries in each column of $M$ is equal to $1$.
%
A probability distribution over the states of a Markov chain $M$ is represented\footnote{To simplify notation, we may represent column vectors using their transposes.}  
 by a column vector, for instance, $v = [1/2, 1/8, 3/8]^T$ (\cref{MChain:fig}).
%
%
This vector specifies that the system is in the states \emph{red}, \emph{black}, and \emph{green} with probabilities $1/2$, $1/8$, and $3/8$, respectively.  
%
To compute the distribution after one transition step, we multiply the transition matrix $M$ by the column vector $v$, yielding a new distribution vector, here, $v' = [9/20, 1/4, 3/10]^T$.


The quantum setting operates over complex numbers rather than real-valued probabilities.  
%
In this context, a transition weight is not a real number $p \in [0,1]$, but a complex number $c$ such that the squared modulus $|c|^2$ lies in the interval $[0,1]$.  
%
This seemingly subtle change has profound implications for system dynamics.  
%
Whereas real probabilities can only accumulate positively under addition, complex amplitudes can interfere—constructively or destructively.  
%
For example, if $p_1, p_2 \in [0,1]$ are real numbers, then both $p_1 + p_2 \geq p_1$ and $p_1 + p_2 \geq p_2$ hold.  
%
In contrast, for complex numbers $c_1, c_2$, it is not necessarily the case that the squared modulus of their sum dominates the individual squared moduli:  
\[
|c_1 + c_2|^2 \leq \max\left(|c_1|^2, |c_2|^2\right)
\]
%
This phenomenon, known as \emph{interference}, is a hallmark of quantum systems.  
%
In particular, destructive interference allows for the cancellation of probability amplitudes, leading to outcomes with lower probabilities than suggested by the individual components.  
%
Such behaviours are central to the analysis and modelling of quantum processes, where unitary evolution replaces the stochastic matrices of classical probabilistic systems.



