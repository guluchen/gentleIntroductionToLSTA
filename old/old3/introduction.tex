\section{Introduction}
\label{introduction:section}
We may get substantial  added value when connecting two complementary areas of computer science.
%
This applies particularly when adapting mature techniques developed in one area to solve complex problems that arise in another (tyupically) new area.
%
The current paper illustrates one such a case. 
%
It describes the application of techniques developed in logic, automata, and symbolic verification to analyze the correctness of quantum circuits.

The current quest of quantum computing is achieving  so called {\it quantum supremacy}, meaning that we reach a stage in the technology development
where we can   solve problems that are practically unsolvable using conventional computing.
%
To that end, substantial effoprt is ongoing to implement quantum hardware and develop programming languages for quantum computers.
%
In many applications, system correctness is of critical importance.
%
For instance, identifying a subtle bug in a quantum circuit used for quantum chemistry simulations could prevent incorrect predictions about molecular interactions, which are critical for drug discovery and material design. 
%



Given the complexity of quantum systems and the above mentioned correctness requirements, tools for analyzing and verifying quantum programs' correctness are of critical importance.
%
However, building verification tools for quantum programs is a formiddable challenge.
%
First,  due their behaviors are probabilistic in nature, and their computational space is exponential in size.     
%
Furthermore, there is a fundamental difference in the manner in which conventional and quantum computers store information: conventional computers use digital bits (0s and 1s), while quantum computers use  quantum bits (qubits) that probabilistically have values in nthe intgerval between 0 and 1.

The current paper reports on research in whose goal is adating the research community's vast experience in conventional program verification to develop tools for verifying quantum systems.
%
Verification tools, in general, and for quantum systems in particular, would ideally have the following properties: (1) Flexibility: allows flexible specification of properties of interest, (2) Diagnostics: provides precise bug diagnostics, (3) Automation:
Operates automatically, and (4) Scalability: scales efficiently to verify useful programs.
%
Symbolic verification is one of the most successful techniques that satisfy the above criteria for conventional programs.
%
Notable instances, are invariant certification (e.g., in the form of Hoare triples), and model checking.
%
Despite the overwheling success achieved in the verification of conventional hardware and software,  we are lacking an extension to the quantum realm.
%
The principal reasonm isthe latter's unique mathematical structure and different operational principles.                          
                                                                               
In this work we describe an innovative application of automata theory to quantum circuit verification.
%
%
More precisley, we  combine we use tree automata-based symbolic representations for represeting quantum states.
%
The class of tree we consider is tailored for quantum circuits, considerably extending the sacalbility of quantum circuit verification comnpared to existing techniques.
%
As expected, when dealing with an entirely new application area, we 
we go back to the very basics of program verification.
%
We consider classical Hoare triples $\{P\} C \{Q\}$, where $P$, the {\it pre-condition}, represents a set of initial quantum states, $Q$, the {\it post-conditions} represents a set of quantum {\it target} states, and $C$ is a quantum circuit.
%
It uses  symbolic verification to check the
validity of the triple, ensuring that all executions of $C$ from states in $P$ result in states within $Q$.
%
More precisely, we will resprent $P$ and $Q$ by a special form of tree automata, and provide algorithms to check whether we reach $Q$ from $P$ if we execute $C$.
%
The framework  is based on the observation that we can lift core concepts of classical verification, such as state-space exploration and symbolic reasoning, into the quantum setting. 



Although this is the first attempt to use such techniques in the context of quantum computing, implementing the framework with a tool gives spectacular results.
%
For instance, the tool manages to verify large circuits with up to 40 qubits and 141,527 gates or catch bugs injected
into circuits with up to 320 qubits and 1,758 gates, while all exiting tools fail to handle such benchmarks.
%

%
This extension allows the verification of quantum circuits, offering a path to applying well-established classical paradigms in a domain where formal guarantees are critical. 




