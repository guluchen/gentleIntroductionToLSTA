\section{Circuits} 

%^

\begin{wrapfigure}{R}{0.5\textwidth}
\includegraphics[width=0.48\textwidth]{Figures/HWCircuits/DigitalCircuit} 
\caption{A classical combinatorial circuit}
\end{wrapfigure}
A combinatorial circuit consists of a set of gates and acts as a Boolean function. It takes a sequence of bits as input and produces a sequence of bits as output. There are also internal bits that represent the connections between gates. In the figure, the circuit has three input bits, one output bit, and three internal bits.

%^

\begin{wrapfigure}{r}{0.5\textwidth}
\includegraphics[width=0.48\textwidth]{Figures/Circuits/HZ} 
\caption{A simple quantum circuit.}
\label{HZ:fig}
\end{wrapfigure}

\cref{HZ:fig} shows a quantum circuit consisting of an $\hgate$ followed by a $\zpgate$ gate.

%^

We start with a qubit in a superposition state created using a Hadamard gate, and then apply a $\zpgate$ gate to it.

Here is a step-by-step description of the circuit behavior when the starting qubit is $\ketof0$.

\begin{enumerate}
\item Start in $\ketof0 = [1 \;\; 0]^T$
\item Apply the Hadamard gate: 
$H\ketof0 = \frac{1}{\sqrt{2}}(\ketof0 + \ketof1)$

Now the qubit is in an equal superposition.
\item Apply the $\zpgate$ gate: 
$\zpgate\left(\frac{1}{\sqrt{2}}(\ketof0 + \ketof1)\right) = \frac{1}{\sqrt{2}}(\ketof0 - \ketof1)$.

The $\ketof1$ term’s sign flips—this is phase inversion.
\end{enumerate}

Intuitively, the Hadamard spreads the probability amplitude across both states.

The $\zpgate$ keeps the probability unchanged (as it only affects the phase), but introduces a phase shift to $\ketof1$, which is crucial in interference patterns for quantum algorithms.

%^

\begin{figure}[ht] 
    \centering
    \includegraphics[scale=1.2]{Figures/BDDs/BDDs} 
    \caption{The BDD representation of a quantum state.}
    \label{BDD:fig}
\end{figure}

%^

Next, we turn our attention to the verification of quantum circuits.

We can formulate the {\it program verification} problem as an instance of classical Hoare Logic:
\[
P \{C\} Q
\]
where $P$ is the {\it pre-condition}, $Q$ is the post-condition, and $C$ is a program.

The pre-condition $P$ characterizes a set of {\it initial} ({\it source}) states, and the post-condition $Q$ characterizes a set of {\it final} ({\it target}) states.

We would like to verify that if we execute $C$ from any state satisfying $P$, then we end up in a state satisfying $Q$.

Here, we take $C$ to be a quantum circuit.

A fundamental breakthrough in the verification of conventional computer systems was the invention of efficient data structures to represent sets of states.

%^

A case in point is the classical BDD (Binary Decision Diagram) data structure.

\cref{BDD:fig} depicts a BDD representation of a quantum state.

The main challenge in using BDDs in quantum circuits is the fact that a BDD represents a single state.

In conventional circuits, a BDD represents a (large) set of states.

Thus, we need a framework in which we can handle sets of BDDs.

Given that we represent quantum states by trees, a natural choice is to use tree automata to represent sets of quantum states.

%^

\begin{figure}[ht] 
    \centering
    \includegraphics[scale=0.8]{Figures/Automata/aut3} 
    \caption{(a) A tree automaton for generating all basis states of size $3$.
      The dashed lines represent the left child, and the bold lines represent the right child.
      (b) The tree corresponding to $[0\;0\;0\;0\;0\;0\;1\;0]^T$. The red rules in the automaton are those used to generate the tree.
      (c) An automaton to generate all basis states of size $n$.}
    \label{automata:fig}
\end{figure}

%^

\cref{automata:fig}(a) gives a tree automaton for accepting all the basis states of size three.

The set of rules is:
\[
\begin{array}{lll}
p \xrightarrow{x_0} (q_0, q_1) &
\;\;\;\;\;\; p \xrightarrow{x_0} (q_1, q_0) & \\
q_0 \xrightarrow{x_1} (r_0, r_0) &
\;\;\;\;\;\; q_1 \xrightarrow{x_1} (r_1, r_0) &
\;\;\;\;\;\; q_1 \xrightarrow{x_1} (r_0, r_1) \\
r_0 \xrightarrow{x_2} (s_0, r_0) &
\;\;\;\;\;\; r_1 \xrightarrow{x_2} (s_1, s_0) &
\;\;\;\;\;\; r_1 \xrightarrow{x_2} (s_0, s_1) \\
s_0 \xrightarrow{} 0 &
\;\;\;\;\;\; s_1 \xrightarrow{} 1
\end{array}
\]

Notice that we are characterizing $2^n$ basis states using only $3n+1$ transitions.

%^

\begin{figure}
\includegraphics{Figures/Circuits/Bells}
\caption{The Bell gates}
\label{Bells:fig}
\end{figure}

In our setting, a gate application corresponds to a tree transformation.

For a circuit consisting of a number of gates, we provide an algorithm that transforms a tree automaton describing a set of input states to a new automaton describing the set of output states.

We do this by constructing a sequence of automata that model the effects of each gate.

\cref{Bells:fig} gives a concrete example to demonstrate our approach.

%^

Assume we want to design a circuit constructing the Bell state, i.e., a 2-qubit circuit converting a basis state $\ketof{00}$ to the maximally entangled state $\frac{1}{\sqrt{2}}(\ketof{00} + \ketof{11})$.

Given the automaton corresponding to the pre-condition (\cref{Bells:fig}(a)), we derive the automaton corresponding to the post-condition (Fig.~1b).

In this case, both automata use $q$ as the root state and accept only one tree.

We can observe the correspondence between quantum states and tree automata by traversing their structures.

%
